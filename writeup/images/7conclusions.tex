\chapter{Conclusions}
\label{chapter:conclusions}
SSA is a comparatively recent statistical cryptanalytic technique among various others, as for example linear and differential cryptanalysis. Researchers have tried and established statistical links among different techniques. Blondeau and Nyberg showed one important link in between TD and SS attacks. They showed that TD attack using structures is identical to the sampling algorithm of SSA \citep{Celine_Kaisa_Links_2014}. Then they used the statistical model of the TD attack to explain the behaviour of SS attack. However, there was no statistical model available which was developed based on the properties of SSA directly. In this thesis, we used the distribution at the output of an SS trail directly to develop a statistical model instead of using any link with other techniques. \par \noindent If an SS trail is chosen wisely as discussed in Chapter \ref{chapter:block_cipher_cryptanalysis}, then there is a significant degree of non-uniformity in the distribution of the values at the output of the trail. %In this thesis we have discussed the principle of choosing a useful SS trail which has significant non-uniformity at the output of the trail. 
As discussed in Chapter \ref{chapter:block_cipher_cryptanalysis}, to perform an SSA, we are in need of an attack that distinguishes the non-uniform distribution from uniform distribution, which eventually can be transformed into a key recovery attack. This thesis has focused only on this distinguishing attack and has not discussed the key recovery attack in detail. In Chapter \ref{chapter:statistics}, we have presented a statistical test that can perform this distinguishing attack. To perform the statistical test %we were in need of a known non-uniform distribution. So, we continued to 
we have developed the statistic $T$ based on the distribution of the values at the output bits of the SS trail, when the bits at the input of the SS trail are fixed and sufficiently many plaintexts are encrypted. The plaintexts differ from each other only in the non-trail input bits. \par \noindent In Chapter \ref{chapter:statistical_distinguishers}, we have derived the distribution of a few different variants of this statistic $T$. We have shown that all of these variants of $T$, which are originally $\chi^2$ distributed are also approximately normally distributed. The mean and variance of all of these variants of $T$ are also derived in this chapter, which enables us to perform the statistical test. We also have derived a reasonable overestimation of the number of required plaintexts (in Chapter \ref{chapter:data_complexity_of_SSA}) to be encrypted to perform the distinguishing attack with an arbitrarily fixed success probability, which is referred as the data complexity of SSA and denoted by $N_{SS}$. Finally, in Chapter \ref{chapter:experiment}, we have verified the statistical model by experimenting on a small variant of the block cipher PRESENT called SMALLPRESENT-[$4$] for the case of a single fixation. The result shows that, the distinguishing attack is successful with very high success probability within the theoretical data complexity bound for smaller rounds. For large number of rounds, the theoretical data complexity is larger than the full code book excluding the fixed bits. As a result they do not distinguish at all using a single fixation. It could be the case that if we used multiple fixations, the distinguisher would distinguish itself from the uniform distribution. That is, there is a scope of more experiments based on multiple fixations.
\par \noindent Both in the theories and experiments it has been considered that, the sample for each fixation is chosen randomly with replacement. When the sample size approaches the full code book excluding the fixed bits, the only sensible option is to use sampling without replacement. However, in real life cryptanalysis the sample size almost never appraoches the full codebook. As a result, we recon, sampling with replacement is good enough for a successfull distinguishing attack. However, in an upcoming paper \citep{kaisa_mohsin_2015},  the case of sampling without replacement is considered. And the experiments have also been extended to SMALLPRESENT-[$8$] for both of the cases of sampling with or without replacement.

%%%%%%%%%%%%%%%%%%%%%%%%%%%%%%%%%%%%%%%%%%%%%%%%%%%%%%%%%%%%%%%%%%%%%%%%%%%%%%%%%%%%%%%%%%%%%%%%%%%%%%%%%%%%%%%%%%%%%%%%%%%%%%
%%%%%%%%%%%%%%%%%%%%%%%%%%%%%%%%%%%%%%%%%%%%%%%%
\iffalse
\par that we used to perform the statistical test. We fixed the bits at the input of the SS trail and encrypted sufficiently many plaintexts which differed from each other only in the non-trail input bits. Then we observed the distribution of the values at the output bits of the trail. We computed a statistic from this distribution which is originally $\chi^2$ distributed. Then we showed that this statistic is approximately a normal deviate and can be used to perform the distinguishing attack. We computed the distribution of a few different variants of this statistic. We also computed a reasonable overestimation of the number of required plaintexts to be encrypted to perform the distinguishing attack with a certain high success probability, which we called to be the data complexity of the attack. We validated this statistical model by experimenting on a small variant of the block cipher PRESENT called SMALLPRESENT-[$4$]. We found that, the distinguishing attack was successful with almost probability $1$ within the theoretical data complexity bound for smaller rounds. We kept our experiment limited for the case of single fixations. For higher rounds, the theoretical data complexity was larger than the full code book excluding the fixed bits. So, it could be the case that if we use multiple fixations, the distinguisher will distinguish itself from the uniform distribution. That is, there is a scope of more experiments based on multiple fixations. \par \noindent Both in our theory and experiment we considered that, the sample for each fixation was chosen randomly with replacement. However, as a future work, it will be very interesting to check how the theory evolves when the sampling is done without replacement. And the experiment could also be extended to SMALLPRESENT-[$8$] for both of the cases of sampling with or without replacement.
\fi